\documentclass{article}
\usepackage{hwopt}
\usepackage{amsmath}	% Package for AMS
\usepackage{amsthm}     % Package for AMS-therom
\usepackage{amssymb}	% Package for AMS-symbol
\usepackage{bm}

\newcommand{\xB}{\bm{x}}
\newcommand{\yB}{\bm{y}}
\newcommand{\gB}{\bm{g}}
\newcommand{\RBB}{\mathbb{R}}
\newcommand{\FM}{\mathcal{F}}
\newcommand{\SM}{\mathcal{S}}
\newcommand{\LM}{\mathcal{L}}
\newcommand{\domf}{\textrm{dom}\;f}
\newcommand{\functiontype}[3]{\FM_{#1}^{#2,#3}(\RBB^n)}
\newcommand{\normgen}[1]{\left\| #1 \right\|}
\newcommand{\strongconvextype}[2]{\SM_{#1}^{#2}(\RBB^n)}
\newcommand{\argmin}{\mathop{\rm argmin}}

%%%%%%%%%%%%%%%%%
%     Title     %
%%%%%%%%%%%%%%%%%
\title{Coursework (5) for \emph{Introductory Lectures on Optimization}}
\author{Hanxuan Li \\ 3220106039}
\date{Dec. 9, 2023}

\begin{document}
\maketitle

\begin{excercise}\label{e0}
Prove the following theorem: \\
for any $\xB_0 \in \domf$, all vectors $\gB \in \partial f(\xB_0)$ are supporting to the level set $\LM_f (f(\xB_0))$:
\begin{equation}
	\innerproduct{\gB}{\xB_0 - \xB} \geq 0, \;\;\; \forall \xB \in \LM_f(f(\xB_0)) \equiv \{ \xB \in \domf: f(\xB) \leq f(\xB_0) \}
\end{equation}
\end{excercise}
\begin{PROOF}{e0}

Since $\gB \in \partial f(\xB_0)$, we have:
\begin{equation}
	\forall \xB \in \domf, f(\xB)\ge f(\xB_0)+\innerproduct{\gB}{\xB-\xB_0}
\end{equation}
Also, we have $\xB \in \LM_f(f(\xB_0))$, which means $f(\xB) \leq f(\xB_0)$. Thus, we have:
\begin{equation}
	f(\xB_0) \geq f(\xB) \geq f(\xB_0) + \innerproduct{\gB}{\xB - \xB_0}
\end{equation}
Thus we have $\innerproduct{\gB}{\xB - \xB_0} \le 0$. That is :
\begin{equation}
	\left\langle\gB,\xB\right\rangle\le\left\langle\gB,\xB_0\right\rangle,\;\;\;\forall \xB\in\LM_f(f(\xB_0))
\end{equation}
So by the definition of the supporting hyperplane, we have $\gB$ is supporting to the level set $\LM_f (f(\xB_0))$.

\end{PROOF}

\begin{excercise}\label{e1}
Prove the following theorem: \\
let $Q \subseteq \domf$ be a closed convex set,  $\xB_0 \in Q$ and 
 \begin{equation}
 	\xB^* = \argmin \{ f(\xB) | \xB \in Q \}
 \end{equation}
 Then for any $g \in \partial f(\xB_0)$ we have $\innerproduct{\gB}{\xB_0 - \xB^*} \geq 0$.
\end{excercise}
\begin{PROOF}{e1}
	
Since $\xB^* = \argmin \{ f(\xB) | \xB \in Q \}$, we have:
\begin{equation}
	\forall \xB \in Q, f(\xB^*) \leq f(\xB)
\end{equation}
Also, we have $\xB_0 \in Q$, which means $f(\xB_0) \geq f(\xB^*)$. Thus, we have:
\begin{equation}
	f(\xB_0) \geq f(\xB^*) \geq f(\xB_0) + \innerproduct{\gB}{\xB^* - \xB_0}
\end{equation}
Thus we have $\innerproduct{\gB}{\xB^* - \xB_0} \le 0$. That is:
For any $\gB \in \partial f(\xB_0)$, we have $\innerproduct{\gB}{\xB_0 - \xB^*} \geq 0$.

\end{PROOF}

\begin{excercise}\label{e2}
Prove the following theorem: \\
let $f$ be closed and convex. Assume that it is differentiable on its domain. Then 
$\partial f(\xB) = \{ \nabla  f(\xB) \}$
for any $\xB \in \textrm{int}(\domf)$.
\end{excercise}

\begin{PROOF}{e2}
Notice that $\forall \bm{p} \in \mathbb{R} ^n$, we have:
\begin{equation}
	\left\langle \nabla f(\xB),\bm{p}\right\rangle = \lim_{t\rightarrow 0} \frac{f(\xB+t\bm{p})-f(\xB)}{t}
\end{equation}
Becausen $f^{'}(\bm{x}_0;\bm{p})=max\left\{\left\langle\bm{g},\bm{p}\right\rangle|\bm{g}\in\partial f(\bm{x}_0)\right\}$ for all $\bm{x}\in int(\domf)$
So we have:
\begin{equation}
	\left\langle \nabla f(\xB),\bm{p}\right\rangle = f^{'}(\bm{x};\bm{p})\ge \left\langle\bm{g},\bm{p}\right\rangle
\end{equation}
That is:$\forall \bm{p}\in\mathbb{R}^n,\left\langle\nabla f(\bm{x})-\bm{g},\bm{p}\right\rangle\ge 0$.
Let $\bm{p}=\bm{g}-\nabla f(\bm{x})$,we have:
\begin{equation}
	\Vert\bm{g}-\nabla f(\bm{x})\Vert^2\le 0
\end{equation}
So we have $\bm{g}=\nabla f(\bm{x})$.
Because of the arbitrariness of $\bm{p}$ thus the arbitrariness of $\bm{g}$, we can reach the conclution that $\partial f(\bm{x})=\left\{\nabla f(\bm{x})\right\}$ for any $\xB \in \textrm{int}(\domf)$.

\end{PROOF}
\end{document}