\documentclass{article}
\usepackage{hwopt}

%%%%%%%%%%%%%%%%%
%     Title     %
%%%%%%%%%%%%%%%%%
\title{Coursework (2) for \emph{Introductory Lectures on Optimization}}
\author{Hanxuan Li \\ 3220106039}
\date{Oct. 16, 2023}

\begin{document}
\maketitle

\begin{excercise}\label{e0}
For the function $f(x): \mathbb{R}^n \rightarrow \mathbb{R}^m$, please write down the zeroth-order Taylor expansion with an integral remainder term.
\end{excercise}

\begin{SOLUTION}{e0}
\begin{equation}
	f(x)=f(a)+\int_0^1\nabla f(a+t(x-a))(x-a)dt
\end{equation}
\end{SOLUTION}

\begin{excercise}\label{e1}
Please write down the definition of the $p$-norm for a $n$-dimensional real vector.
\end{excercise}

\begin{SOLUTION}{e1}
\begin{equation}
	\Vert x \Vert_p=(\vert x_1\vert^p+\vert x_2\vert^p+\cdots+\vert x_n\vert^p)^{\frac{1}{p}}
\end{equation}
\end{SOLUTION}


\begin{excercise}\label{e2}
Please write down the definition of the matrix norms induced by vector $p$-norms.
\end{excercise}

\begin{SOLUTION}{e2}
\begin{equation}
	\Vert A \Vert_p=\mathop{sup}_{x\neq0}\frac{\Vert Ax\Vert_p}{\Vert x\Vert_p}
\end{equation}
\end{SOLUTION}


\begin{excercise}\label{e3}
	Let $A$ be an $n \times n$ symmetric matrix. Proof that $A$ is positive semidefinite if and only if all eigenvalues of $A$ are nonnegative. Moreover, $A$ is positive definite if and only if all eigenvalues of $A$ are positive.
\end{excercise}

\begin{PROOF}{e3}
	\begin{enumerate}
		\item $A$ is positive semidefinite iff all eigenvalues of $A$ are nonnegative
		\begin{enumerate}
			\item Sufficient condition:
			If $A$ is semidefinite, then for any non-zero vector $x$, we have $x^TAx\ge0$. \\
			Let $\lambda$ be an arbitary eigenvalue of $A$, and $v$ be the corresponding eigenvector, thus we have $Av=\lambda v$.\\
			Then:
			\begin{equation}
				v^TAv=v^T\lambda v=\lambda \Vert v\Vert^2 \ge 0
			\end{equation}
			Since $\Vert x\Vert \ge0$, it follows that $\lambda$ must be nonnegative.\\
			Without loss of generality, all eigenvalues of $A$ are nonnegative.
			\item Necessary condition:
			Since $A$ is a symmetric matrix, so it must have $n$ eigenvalues,let's say: $\lambda_1,\lambda_2,\cdots,\lambda_n$, and the corresponding eigenvectors(normalized,i.e.$\Vert v_i\Vert==1$) $v_1,v_2,\cdots,v_n$ which are linearly independent and actually orthogonal to each other, that is:
			\begin{equation}
				v_i^Tv_j=
				\begin{cases}
					0&i\neq j\\
					1&i=j
				\end{cases}
			\end{equation}
			So $\forall x$,we can express $x=c_1v_1+c_2v_2\cdots c_nv_n$, where $c_1,c_2,\cdots,c_n \in \mathbb{R}$. \\ Then we have:
			\begin{equation}
				x^TAx=(c_1v_1+c_2v_2\cdots c_nv_n)^TA(c_1v_1+c_2v_2\cdots c_nv_n)=c_1^2\lambda_1+c_2^2\lambda_2+\cdots+c_n^2\lambda_n \ge 0
			\end{equation}
			So $A$ is positive semidefinite.
		\end{enumerate}
		\item $A$ is positive definite iff all eigenvalues of $A$ are positive.\\
		The proof of this statement is almost the same to the previous one, with the only modification being changing ``greater than or equal to" sign to ``strict greater" sign.
	\end{enumerate}
\end{PROOF}

\end{document}